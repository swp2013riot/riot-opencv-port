\documentclass[10pt,a4paper]{article}

\usepackage[utf8]{inputenc}
\usepackage[german]{babel}
\usepackage[pdfborder={0 0 0}]{hyperref}
\usepackage{tabularx}

\setlength{\parindent}{0cm}

\title{Freie Universität Berlin \\
	Institut für Informatik \\
	SoSe 2013 \\ \ \\
	Softwareprojekt Telematik: \\ \ \\
	\textbf {Portierung von OpenCV auf RIOT OS} \\
	\textbf {am Beispiel von BackgroundSubtractorMOG} \\ \ \\
	Projektbericht \\ \ \\}
\author{Daniel Akrap  \textless\href{mailto:daniel.akrap@fu-berlin.de}{daniel.akrap@fu-berlin.de}\textgreater
        \and Lidia Krus \textless\href{mailto:ruskrus@gmail.com}{ruskrus@gmail.com}\textgreater 		\\ \\
	Betreuerin: Dipl.-Inform. Alexandra Danilkina \\ 
	\textless\href{mailto:alexandra.danilkina@fu-berlin.de}{alexandra.danilkina@fu-berlin.de}\textgreater}
\date{09. Juli 2013}

\begin{document}

\maketitle

\newpage
\section*{Abstract}

Das Ziel des Softwareprojektes ist die Erschließung der Möglichkeiten für die Portierung von OpenCV auf das Betriebssystem RIOT mit Bezug auf Projekt SAFEST, dessen Schwerpunkt auf der Verbesserung der Sicherheit an Flughäfen mittels Überwachung von Menschenansammlungen liegt. Die Portierung von OpenCV auf RIOT soll Bildverarbeitung in Echtzeit auf der Plattform RIOT ermöglichen und damit die Möglichkeiten im Bereich des Internet of Things erweitern. Als Beispielanwendung wird ein Programm entwickelt, das Hintergrundentfernung bei der Bildverarbeitung macht.

\newpage
\tableofcontents
\setcounter{tocdepth}{3}

\newpage
\section{Einleitung}

Dieses Dokument ist der Schlussbericht des Softwareprojekts \glqq Portierung von OpenCV auf RIOT OS\grqq. Das Projekt wurde vom 23.04.2013 bis 09.07.2013 von Daniel Akrap und Lidia Krus unter Betreuung von Dipl.-Inform. Alexandra Danilkina im Rahmen der Lehrveranstaltung Softwareprojekt Telematik am Institut für Informatik an der FU Berlin durchgeführt. \\

Der Projektbericht beschreibt die Aufgabenstellung, die Planung, das Umfeld sowie den Ablauf vom Projekt. Des Weiteren werden die erzielten Ergebnisse ausführlich beschrieben sowie Einsatzempfehlungen für die entwickelte Lösung gegeben. Schließlich werden der Nutzen und die Verwertbarkeit der Ergebnisse ausgeführt.

\subsection{Aufgabenstellung}

Der Schwerpunkt dieses Softwareprojekts liegt auf der Erschließung der Möglich-keiten zur Portierung von OpenCV auf das Betriebssystem RIOT. \\

Die vorliegende Arbeit hat Bezug zum Projekt SAFEST (Social-Area Framework for Early Security Triggers at Airports). Das ist ein deutsch-französisches Gemeinschaftsprojekt zur Überwachung von Menschenansammlungen, die der Verbesserung der Sicherheit an Flughäfen dient. Die Daten zur Überwachung von von Menschenansammlungen werden mit Videokameras aufgenommen und sollen mit der Graphikbibliothek OpenCV (Open Source Computer Vision Library) verarbeitet werden. Diese Bibliothek enthält Algorithmen für Bildverarbeitung und Maschinelles Sehen mit Fokus auf Echtzeitfähigkeit. \\

Im Rahmen des Softwareprojekts Telematik wird ein Versuch unternommen, diese Bibliothek auf das Betriebssystem RIOT OS zu portieren. Das Mikrokernel-Betriebssystem RIOT soll unterschiedlichsten Software-Anforderungen an die Geräte im Internet of Things (IoT) entsprechen [Baccelli et al. 2012: 3]. RIOT OS ist modular aufgebaut, unterstützt Multithreading und ist echtzeitfähig. \\

Die Portierung von OpenCV auf RIOT ermöglicht Bildverarbeitung in Echtzeit auf der Plattform RIOT und erweitert damit die Möglichkeiten im IoT-Bereich. \\

Da im Projekt SAFEST der erste Bildverarbeitungsschritt auf die Vorverarbeitung auf der Node mittels Separierung von Personen und Menschenansammlungen von dem Hintergrund zwecks Verringerung des Datenumfangs ausgerichtet ist, haben wir uns für den folgenden Anwendungsfall als Vorgabe für unsere Testanwendung entschieden: 

\begin{itemize}
\item Das Programm bekommt eine Reihe von Bildern als Eingabe.
\item Diese Bildern werden an die Funktion zur Hintergrundentfernung übergeben. 
\item Die Funktion berechnet Hinter- und Vordergrund und aktualisiert das Hintergrundmodell.
\item Die Funktion erzeugt eine neue Bilddatei als Ergebnis der Hintergrundentfernung am letzten Bild, das es bekommen hat. Diejenigen Pixel, die zum Hintergrund gehören, sind auf diesem Bild schwarz dargestellt, die Vordergrundpixel bekommen weiße Farbe.
\end{itemize}

Dieser Anwendungsfall setzt die Portierung der Funktionalität zur Hintergrundentfernung aus dem für das Projekt SAFEST relevanten OpenCV-Modul {\it video} voraus.

\subsection{Stand der Technik}

In unserem Projekt haben wir mit OpenCV (Version 2.4.5) und RIOT OS (Stand April-Mai 2013) gearbeitet.  \\

Programmentwicklung unter RIOT OS erfolgt mit den Standardentwicklungswerkzeugen wie gcc, gdb in den Programmiersprachen C oder C++. Das Betriebssystem wird aktiv weiter entwickelt und wird regelmäßig aktualisiert. \\

OpenCV ist in C und (ab Version 2.0) in C++ implementiert.

\subsection{Projektplanung}

Die Arbeit an dem Projekt umfasste die Tätigkeiten zum Management, der Dokumentation und der Außendarstellung des Projekts mittels 3 Präsentationen am Anfang, in der Mitte und am Ende des Projekts. Um den erfolgreichen Ablauf zu sichern, wurden in der Anfangsphase Meilensteine definiert und ein Arbeitsplan erarbeitet, der sich jedoch während der Projektentwicklung erheblich geändert hat. 

Wir haben das Projekt in vier Teilprobleme gegliedert. Diese sind:
\begin{itemize}
\item Einrichtung der Arbeitsumgebung und Einarbeitung in die Projekte RIOT OS und OpenCV.
\item Test auf Machbarkeit der Portierung von OpenCV (Feststellung des Umfangs, in dem Programme in C++ von RIOT OS unterstützt sind).
\item Entwicklung einer auf RIOT OS lauffähigen Musteranwendung für Bildverarbeitung
\item Erstellung einer erweiternden Dokumentation für die Benutzung von OpenCV bzw. C++ unter RIOT
\end{itemize}

Dabei haben wir uns an folgende Meilensteine orientiert: \newline
1) Ein c++ Programm ist lauffähig unter RIOT; \newline
2) Einlesen und Schreiben der (Bild)Dateien ist unter RIOT möglich; \newline
3) Unterstützung der benutzten Datentypen ist sichergestellt; \newline
4) Anwendung zur Hintergrundentfernung ist erstellt. \newline

Für die Zusammenarbeit am Projekt und die Versionsverwaltung haben wir das Plattform Github benutzt. Außerdem fanden wöchentliche Treffen der Teammitglieder statt. \\


\newpage
\section{Entwicklung der Aufgabenstellung}

Wir haben den Schwerpunkt unseres Projektes darauf gelegt, die potentiellen Möglichkeiten für das Zusammenspiel von OpenCV und RIOT OS zu untersuchen. Wir sind davon ausgegangen, dass die Portierung erfolgreich sein würde, da zumindest theoretische Voraussetzungen dafür gegeben waren: Einerseits, ist OpenCV eine modulare Graphik-Bibliothek mit Fokus auf C++; andererseits, unterstützt RIOT OS die Entwicklung sowohl in C, als auch in C++. Wir haben uns zum Ziel gesetzt, diese theoretische Annahme in der Praxis zu überprüfen. \newline

Während der Einarbeitungszeit hatten wir viele Schwierigkeiten aufgrund mangelhafter Dokumentation und öfteren Veränderungen am RIOT. Viel Zeit hat der Versuch gekostet, ein einfaches C++-Programm unter RIOT zum Laufen zu bringen bzw. den Grund für Probleme damit herauszufinden. Letztendlich hat sich herausgestellt, dass die Unterstützung vieler grundlegender C++-Bibliotheken fehlt (wie <fstream>, <iostream> usw.), die z.B. mit Systemzugriff arbeiten, was für Portierbarkeit von OpenCV entscheidend wäre. \newline

Dadurch haben sich technische Anforderungen geändert und es wurde entschieden, mindestens zu versuchen, einen Beispielalgorithmus von OpenCV (BackgroundSubtractorMOG) in einer Musteranwendung einzusetzen. Als Nebenleistung sollte eine Hilfsdokumentation zur Benutzung von C++-Programmen unter RIOT erstellt werden.

\subsection{Herausforderungen}

a) Portierungsprobleme seitens RIOT \newline

Wie es schon oben erwähnt wurde, haben wir eine Reihe der Faktoren, die eine erfolgreiche Portierung von OpenCV auf RIOT OS verhindern, festgestellt:

\begin{itemize}
\item Als IST-Zustand zu Projektbeginn war die C++-Unterstützung seitens RIOT OS mangelhaft.
\item Im Laufe der Entwicklungen zeitgleich zum Projekt wurde Verbesserung der Unterstützung von Nativem Port, viele Fehlerbehebungen usw. beobachtet. Außerdem wurde die Unterstützung von C++ seitens RIOT verbessert (der RIOT-Compiler hat angefangen, nicht nur nach einem main.c, sondern auch nach einem main.cpp Programm zu suchen). 
\item Unterstützung von der Standardbibliothek von C++ blieb aber unkomplett (z.B. wird Input/Output Library nicht eingebunden).
\item Ein gewisses Problem stellten auch verschiedene Toolchain- und Bibliotheksversionen für die Zielplattformen dar.
\end{itemize}

Wür möchten unsere Behauptung, dass für eine erfolgreiche Portierung von OpenCV auf RIOT OS eine komplettere Unterstützung von C++ nötig wäre, mit einigen von uns gefundenen Abhängigkeiten untermauern. \newline

\begin{center}
  \begin{tabular}{ | p{10cm} | }
    \hline
    Nicht unterstützte C++ Bibliotheken \\ \hline

algorithm; array; atomic; bitset; condition\_variable; complex; \newline
forward\_list; fstream; functional; future; iomanip; ios; iostream; \newline 
istream; iterator; locale; map; memory; mutex; ostream; random; \newline
regex; sstream; stdexcept; streambuf; string; system\_error; \newline
thread; tuple; unordered\_map; unordered\_set; valarray \\

    \hline
  \end{tabular}
\end{center}

Beispiele für Abhängigkeiten zwischen OpenCV-Modulen und obengenannten Bibliotheken: \\

{\bf algorithm; array; atomic; bitset:} \newline
opencv-2.4.5/modules/features2d/src/freak.cpp \newline

{\bf condition\_variable; complex:} \newline
opencv-2.4.5/modules/core/include/opencv2/core/core.hpp \newline
opencv-2.4.5/3rdparty/openexr/Imath/ImathRoots.h \newline

{\bf forward\_list; fstream:} \newline
opencv-2.4.5/modules/nonfree/test/test\_detectors.cpp \newline
opencv-2.4.5/modules/ocl/src/initialization.cpp \newline
opencv-2.4.5/modules/ocl/test/precomp.hpp \newline
opencv-2.4.5/modules/gpu/test/test\_precomp.hpp \newline
opencv-2.4.5/modules/gpu/test/nvidia/NCVTest.hpp \newline
opencv-2.4.5/modules/core/test/test\_operations.cpp \newline
opencv-2.4.5/modules/contrib/src/spinimages.cpp \newline
opencv-2.4.5/modules/contrib/doc/facerec/src/facerec\_video.cpp \newline
opencv-2.4.5/modules/contrib/doc/facerec/src/facerec\_fisherfaces.cpp \newline
opencv-2.4.5/modules/contrib/doc/facerec/src/facerec\_save\_load.cpp \newline
opencv-2.4.5/modules/contrib/doc/facerec/src/facerec\_eigenfaces.cpp \newline
opencv-2.4.5/modules/contrib/doc/facerec/src/facerec\_demo.cpp \newline
opencv-2.4.5/modules/contrib/doc/facerec/src/facerec\_lbph.cpp \newline
opencv-2.4.5/modules/calib3d/src/circlesgrid.hpp \newline
opencv-2.4.5/modules/calib3d/test/test\_affine3d\_estimator.cpp \newline
opencv-2.4.5/modules/ml/test/test\_gbttest.cpp \newline
opencv-2.4.5/modules/ml/test/test\_save\_load.cpp \newline
opencv-2.4.5/modules/legacy/src/calonder.cpp \newline
opencv-2.4.5/modules/legacy/test/test\_optflow.cpp \newline
opencv-2.4.5/modules/video/test/test\_tvl1optflow.cpp \newline
opencv-2.4.5/modules/video/test/test\_estimaterigid.cpp \newline
opencv-2.4.5/modules/features2d/src/freak.cpp \newline
opencv-2.4.5/modules/features2d/src/brisk.cpp \newline
opencv-2.4.5/samples/ocl/hog.cpp \newline
opencv-2.4.5/samples/gpu/optical\_flow.cpp \newline
opencv-2.4.5/samples/gpu/hog.cpp \newline
opencv-2.4.5/samples/c/find\_obj\_calonder.cpp \newline
opencv-2.4.5/samples/cpp/pca.cpp \newline
opencv-2.4.5/samples/cpp/matching\_to\_many\_images.cpp \newline
opencv-2.4.5/samples/cpp/videostab.cpp \newline
opencv-2.4.5/samples/cpp/convexhull.cpp \newline
opencv-2.4.5/samples/cpp/bagofwords\_classification.cpp \newline
opencv-2.4.5/samples/cpp/detector\_descriptor\_evaluation.cpp \newline
opencv-2.4.5/samples/cpp/facerec\_demo.cpp \newline
opencv-2.4.5/samples/cpp/stitching.cpp \newline
opencv-2.4.5/samples/cpp/detector\_descriptor\_matcher\_evaluation.cpp \newline
opencv-2.4.5/samples/cpp/tvl1\_optical\_flow.cpp \newline
opencv-2.4.5/samples/cpp/stitching\_detailed.cpp \newline
opencv-2.4.5/apps/traincascade/imagestorage.cpp \newline
opencv-2.4.5/3rdparty/openexr/IlmImf/ImfTiledOutputFile.cpp \newline
opencv-2.4.5/3rdparty/openexr/IlmImf/ImfStdIO.h \newline
opencv-2.4.5/3rdparty/openexr/IlmImf/ImfOutputFile.cpp \newline
opencv-2.4.5/3rdparty/openexr/IlmImf/ImfInputFile.cpp \newline
opencv-2.4.5/3rdparty/openexr/IlmImf/ImfInputFile.h \newline
opencv-2.4.5/android/service/engine/jni/BinderComponent/ProcReader.cpp \newline

{\bf functional:} \newline
opencv-2.4.5/modules/ocl/src/brute\_force\_matcher.cpp \newline
opencv-2.4.5/modules/imgproc/src/generalized\_hough.cpp \newline
opencv-2.4.5/modules/gpu/include/opencv2/gpu/device/functional.hpp \newline
opencv-2.4.5/modules/gpu/src/precomp.hpp \newline
opencv-2.4.5/modules/gpu/test/test\_precomp.hpp \newline
opencv-2.4.5/modules/stitching/src/precomp.hpp \newline
opencv-2.4.5/modules/contrib/src/spinimages.cpp \newline
opencv-2.4.5/modules/calib3d/test/test\_affine3d\_estimator.cpp \newline
opencv-2.4.5/modules/calib3d/test/test\_chesscorners.cpp \newline
opencv-2.4.5/modules/highgui/src/window\_w32.cpp \newline
opencv-2.4.5/samples/cpp/bagofwords\_classification.cpp \\

Auch dieser kleine Teil an Abhängigkeiten macht deutlich, dass eine komplettere Unterstützung von C++ ein wichtiges Kriterium für die Portierung wäre, bzw. dass unter den gegebenen Bedingungen OpenCV sich nicht auf RIOT OS portieren lässt. Portieren lassen sich einzelne Algorithmen (die keine Systemaufrufe nutzen). \newline

Nach dieser Feststellung mussten wir uns zwischen 2 Wegen für die Weiterarbeit an dem Projekt entscheiden: \newline
1) Entweder wird das Betriebssystem RIOT OS dazu gebracht, das erforderliche Maß an C++-Unterstützung zu erreichen; \newline
2) Oder wir versuchen, einzelne Algorithmen zu portieren bzw. sie anzupassen. Eine solche Anpassung wäre z.B. durch das Lösen der Probleme mit Systemaufrufen möglich, indem die benutzten C++-Aufrufe durch die entsprechenden Funktionalitäten in C ersetzt werden.\newline

Wir haben uns aufgrund des zeitlichen Projektrahmens für die zweite Alternative entschieden. \\

a) Portierungsprobleme seitens OpenCV \newline

Als einen weiteren einschränkenden Faktor haben wir eine große Anzahl an OpenCV-eigenen Datentypen festgestellt, die bei dem Importieren von OpenCV-Modulen durchaus erwünscht sind, weil sie intern für Effizienz und Abgestimmtheit sorgen; bei dem Herausnehmen und Anpassen einzelner Algorithmen aber ein gewisses Problem darstellen, da ihre Äquivalente und Abhängigkeiten untereinander explizit programmiert werden müssen.

\subsection{Folgeentscheidungen}

Als Ergebnis der obenbeschriebenen Untersuchungen haben sich die technischen Anforderungen unseres Projektes geändert und der vorgenommene Portierungsumfang war darauf reduziert, mindestens eine Funktionalität von OpenCV auf RIOT OS in einer Musteranwendung bereitzustellen. Dafür wurde die Funktionalität zur Hintergrundentfernung BackgroundSubtractorMOG aus dem video-Modul von OpenCV gewählt. Außerdem sollte eine Hilfsdokumentation für die RIOT-Wiki erstellt werden.

\newpage
\section{Musteranwendung für RIOT OS}

\subsection{Hintergrundentfernung: Motivation}

Im Prozess des maschinellen Sehens ist Bildsegmentierung üblicherweise der erste Schritt der Bildanalyse. Unter Segmentierung versteht man die Erzeugung von ihnaltlich zusammengehörenden Regionen durch Zusammenfassung benachbarter Pixel entsprechend einem Homogenitätskritärium. Ein typischer Anwendungsfall dafür ist Szenenanalyse in der Videoüberwachung, wobei Segmentierung in Echtzeit erfolgt. Eine der grundlegenden Methoden ist dabei Subtraktion des Hintergrunds. Es wurden viele Hintergrundmodelle entwickelt, die zur Lösung verschiedener Problemstellungen dienen. Ihnen ist gemeinsam, dass dabei angenommen wird, dass die Szenen bzw. Bildregionen ohne eindringende Objekte ein bestimmtes regelmäßiges Verhalten aufweisen, das mit einem statistischen Modell beschrieben werden kann. Eindringende Objekte können dementsprechend durch die Erkundung der in dieses statistische Modell nicht reinpassenden Bildteile bestimmt werden. \\

Da in dem für uns relevanten Projekt SAFEST Hintergrundentfernung einen großen Vorteil der Reduzierung des Speicherverbrauchs bei der Verarbeitung der Daten der Videoüberwachung bringen könnte; und aufgrund der mangelhaften Unterstützung vieler C++-Bibliotheken unter RIOT, was die komplette Portierung erheblich erschwerte, - haben wir für das Herausnehmen und Anpassen des OpenCV-Algorithmus zur Entfernung des Hintergrunds entschieden. Dafür war das Kennenlernen dieses Algorithmus erforderlich. 

\subsection{Algorithmus zur Hintergrundentfernung}

Das von uns gewählte Algorithmus BackgroundSubtractorMOG von OpenCV basiert auf dem theoretischen Modell, das in der Arbeit {\it An Improved Adaptive Background Mixture Model for Real-Time Tracking with Shadow Detection} [KaewTraKulPong, Bowden 2001] vorgestellt ist. \\

Dieses Modell nutzt den pixel-basierten Ansatz zur Hintergrundentfernung: für jedes einzelne Pixel wird eine Wahrscheinlichkeitsdichtefunktion erstellt, die zur Entscheidung herangezogen wird, ob der gegebene Pixel zum Hinter- oder zum Vordergrund gehört. \\

Die Implementierung von OpenCV unterscheidet sich von diesem theoretischen Modell dadurch, dass BackgroundSubtractorMOG keine Schattenerkennung macht. 

\subsection{Unsere Implementierung}

Unser Quellcode umfasst 5 Dateien: {\it image.h, image.cpp, bgfg\_gaussmix.h,

 bgfg\_gaussmix.cpp, main.cpp} sowie Makefiles für das Testen unter Linux bzw. unter RIOT OS. Wir importieren eine geringe Anzahl an den seitens RIOT unterstützten Bibliotheken und zwar: {\it stdio.h, stdlib.h, float.h, math.h, jpeglib.h, list, algorithm}. \\

a) Implementierung nötiger Operationen mit Dateien und Bildern \\

In der Datei image.cpp werden die grundlegenden Funktionalitäten zum Lesen und Schreiben von .jpeg-Dateien; zum Erstellen, Belegen und Befreien der Bildmatrizen; zum Konvertieren zwischen Bildern und Bildmatrizen; sowie zum Erfassen von Bilddimensionen implementiert. Image.h liefert Schnittstelle dafür.

b) Implementierung der Funktionalität zur Hintergrundentfernung \\

Hier wird vor allem die Funktion {\it process8uC1()} aus dem BackgroundSubtractorMOG an unsere Anwendung angepasst. Sie stellt die Möglichkeit zur Hintergrundentfernung an den schwarz-weißen 8-bit Bildern bereit. Entfernung des Hintergrunds geschieht analog zu den Anweisungen im OpenCV-Algorithmus, die vorgenommenen Änderungen betreffen nur die Realisierung und nicht die Logik des Algorithmus. \\

c) Testfall \\

Die Datei main.cpp mit dem entsprechenden Makefile wurden zum Testen des Ablaufs der Hintergrundentfernung auf dem Plattform RIOT entwickelt. Entsprechend dem am Anfang postulierten Anwendungsfall werden beim Testen nach dem Dateieinlesen 6 Bildmatritzen erzeugt, die einen grauen Viereck mit einem schwarzen Streifen darstellen. Der Streifen befindet sich auf jedem Bild an einer anderen Stelle und soll als dynamischer Vordergrund interpretiert werden. Die graue Fläche stellt Hintergrund dar. Diese Bildmatritzen und sonstige nötigen Parameter wie Hintergrundmodell, Lernrate, Bilddimensionen, Varianz-Schwellenwert u.a. werden von der Funktion {\it process8uC1()} abgearbeiten und das Ergebnis der Hintergrundentfernung auf dem letzten Bild wird in der Datei {\it fg\_image.jpg} gespeichert. Das Ergebnisbild ist schwarz und hat einen weißen Streifen an der Position des schwarzen Streifens aus dem letzten interpretierten Bild. Das zeugt davon, dass unser Programm die Hintergrundentfernung richtig gemacht hat: der Schwarzbereich gehört zum Hintergrund, das Objekt - der Streifen - gehört zum Vordergrund und wird weiß markiert.

\newpage
\section{Zusammenfassung und Ausblick}

Im Laufe der Projektentwicklung haben wir folgende Ergebnisse bekommen:

\begin{itemize}
\item C++ wird unter RIOT (Stand: Mai 2013) nicht komplett unterstützt. Die aufgedeckten Probleme haben wir in einem Beitrag zur RIOT-Wiki beschrieben. 
\item Dadurch sind aktuell die Möglichkeiten für Zusammenarbeit von OpenCV und RIOT OS als ziemlich begrenzt zu bezeichnen. 
\item Ein entscheidender Kriterium für die Portierbarkeit wäre eine vollständigere Unterstützung von C++ seitens RIOT (insbesondere von Systemzugriffsbibliotheken), was durchaus möglich scheint, da RIOT OS ständig weiterentwickelt und aktualisiert wird.
\item Mittlerweile ist Extraktion einzelner OpenCV-Algorithmen möglich, was allerdings ein ziemlich aufwendiger Prozess ist.
\item Wir haben eine Beispielanwendung entwickelt, das Hintergrundentfernung auf schwarz-weißen *.jpg Bildern unter RIOT OS erlaubt.
\end{itemize}

Für das weitere Vorgehen sehen wir folgende Alternativen: \\

\begin{itemize}
\item Für kleinere (Test)Anwendungen könnte die von uns bereitgestellte Funktionalität zur Hintergrundentfernung benutzt werden. Nach Bedarf könnte sie um Unterstützung farbiger Bilder, anderer Bildformate oder Videosequenzen erweitert werden. Außerdem könnte sie als ein Muster für Extraktion weiterer Funktionalitäten von OpenCV benutzt werden.
\item Für größere Projekte wie SAFEST wäre ein solcher Ansatz unserer Meinung nach nicht erwünscht, das er den Effizienzanforderungen wenig entspricht. Dafür wäre eine erfolgreiche Portierung von OpenCV-Modulen besser geeignet, die wir für möglich halten, wenn die Unterstützung von C++ seitens RIOT OS verbessert wird. Dafür könnte z.B. unsere Beschreibung der entdeckten Probleme im Beitrag zur RIOT-Wiki herangezogen werden. 
\end{itemize}

Im Allgemeinen hat für die Erfahrung der Arbeit an einem Softwareprojekt als positiv und wertvoll erwiesen. Wir haben wichtige softwaretechnische Aspekte kennengelernt: Anforderungenermittlung und -änderungen, Arbeit im Team und Zeitmanagement (mit Gantt-Diagrammen), Projektentwurf und -vorstellung, Versionsverwaltung (mit Github) und Dokumentation. Außerdem haben wir unsere Erfahrungen in der Programmierung in ANSI C und C++ deutlich erweitern können und haben neue Programmierwerkzeuge wie z.B. CMake kennen gelernt. Wir haben Kenntnisse in Bereich der Bildverarbeitung (insbesondere Segmentierung von Vordergrund und Hintergrund) bekommen und ein neues Betriebssystem für IoT kennengelernt. \\

\newpage
\section{Anhang}

\subsection{Quellcode}

// Todo: Hier Quellcode reinschreiben oder einfach nur als zip-Datei abgeben????

\subsection{Glossar}

// Todo: Maybe define some specific termini ???

gaussian mixtures \\
background subtraction \\
background \\
foreground \\


\newpage
\section{Quellen}

\subsection*{Homepages der betreffenden Projekte und Ressourcen}
\begin{itemize}
\item Projekt SAFEST (\textless\href{http://safest.realmv6.org/}{http://safest.realmv6.org/}\textgreater)
\item OpenCV (\textless\href{http://opencv.org}{http://opencv.org}\textgreater)
\item OS RIOT (\textless\href{http://www.riot-os.org}{http://www.riot-os.org}\textgreater)
\end{itemize}

\subsection*{Veröffentlichungen zu den betreffenden Projekten} 
\begin{itemize}

\item KaewTraKulPong P., Bowden R., An Improved Adaptive Background Mixture Model for Real-time Tracking with Shadow Detection. In Proc. 2nd European Workshop on Advanced Video Based Surveillance Systems, AVBS01, Sept 2001. \newline (\textless\href{http://personal.ee.surrey.ac.uk/Personal/R.Bowden/publications/avbs01/avbs01.pdf}{http://personal.ee.surrey.ac.uk/Personal/R.Bowden/publications/avbs01/avbs01.pdf}\textgreater)

\item Baccelli E., Hahm O., Wählisch M., Günes M., Schmidt T., RIOT: One OS to Rule Them All in the IoT. INRIA, Research Report, No. RR-8176, Dec. 2012. \newline (\textless\href{http://hal.inria.fr/hal-00768685/}{http://hal.inria.fr/hal-00768685/}\textgreater)

\end{itemize}

\subsection*{Unser Quellcode und Dokumentierung}
\begin{itemize}
\item SWP 2013: Portierung von OpenCV auf RIOT OS \newline (\textless\href{https://github.com/swp2013riot/riot-opencv-port}{https://github.com/swp2013riot/riot-opencv-port}\textgreater)

\end{itemize}

\end{document}
